\documentclass[conference]{IEEEtran}
\usepackage[utf8]{inputenc}
\usepackage{graphicx}
\usepackage{amsmath}
\usepackage{booktabs}

\title{Fetal Head Circumference Prediction from Ultrasound Images}
\author{22AB13260 - Tran Huy Quan}

\begin{document}

\maketitle

% ============================================
\section{Introduction}
% ============================================

In this lab, I build a model to predict the head circumference (HC) of a fetus from ultrasound images. This is an important measurement that helps doctors monitor fetal development.

\textbf{Objective:} Build a deep learning model to predict HC (in mm) with accuracy measured by MAE (Mean Absolute Error).

% ============================================
\section{Dataset}
% ============================================

\subsection{Data Description}
The dataset consists of fetal ultrasound images with additional information:

\begin{itemize}
    \item \textbf{Training set:} 999 images with HC labels
    \item \textbf{Test set:} 335 images to predict
    \item \textbf{Additional info:} Pixel size (mm) for each image
\end{itemize}

\begin{table}[h]
\centering
\caption{Head Circumference Statistics (mm)}
\begin{tabular}{lr}
\toprule
Metric & Value \\
\midrule
Minimum & 44.3 \\
Mean & 166.2 \\
Maximum & 346.4 \\
\bottomrule
\end{tabular}
\end{table}

\subsection{Data Distribution}
Figure 1 shows the distribution of HC and pixel size in the training set. HC is mostly concentrated in the 150-200 mm range, corresponding to mid-pregnancy.

\begin{figure}[h]
\centering
\includegraphics[width=0.48\textwidth]{figure/fig_distribution}
\caption{Distribution of HC and Pixel Size in the training set}
\end{figure}

\subsection{Sample Images}
Figure 2 displays sample ultrasound images with different HC values. We can see that as HC increases, the head circle in the image becomes visibly larger.

\begin{figure}[h]
\centering
\includegraphics[width=0.48\textwidth]{figure/fig_samples}
\caption{Sample ultrasound images with HC values from 44mm to 178mm}
\end{figure}

% ============================================
\section{Methods}
% ============================================

\subsection{Data Preprocessing}
Images are processed as follows:
\begin{itemize}
    \item Convert to grayscale
    \item Resize to 128×128 pixels
    \item Normalize pixel values to [0, 1]
    \item Split data: 80\% training, 20\% validation
\end{itemize}

\subsection{Model Architecture}
I use a CNN (Convolutional Neural Network) with the following structure:

\begin{enumerate}
    \item \textbf{Conv layer 1:} 32 filters, size 3×3
    \item \textbf{Conv layer 2:} 64 filters, size 3×3
    \item \textbf{Conv layer 3:} 128 filters, size 3×3
    \item \textbf{Global Average Pooling:} Reduce feature size
    \item \textbf{Dense layer:} 64 units + 30\% Dropout
    \item \textbf{Concatenate with Pixel Size} 
    \item \textbf{Dense layer:} 32 units
    \item \textbf{Output:} 1 value (predicted HC)
\end{enumerate}

The model has approximately 130,000 parameters.

\subsection{Training}
\begin{itemize}
    \item \textbf{Loss function:} MAE (Mean Absolute Error)
    \item \textbf{Optimizer:} Adam
    \item \textbf{Epochs:} 50 (early stopping if no improvement)
    \item \textbf{Batch size:} 32
\end{itemize}

% ============================================
\section{Results}
% ============================================

\subsection{Training Process}
Figure 3 shows that MAE decreases over training time. The model stopped at epoch 43 due to no further improvement.

\begin{figure}[h]
\centering
\includegraphics[width=0.45\textwidth]{figure/fig_training}
\caption{MAE on training and validation sets over epochs}
\end{figure}

\subsection{Validation Results}

\begin{table}[h]
\centering
\caption{Evaluation Results}
\begin{tabular}{lr}
\toprule
Metric & Value \\
\midrule
Validation MAE & \textbf{48.79 mm} \\
Validation samples & 200 \\
\bottomrule
\end{tabular}
\end{table}

The average error is about 49mm, meaning the model predicts reasonably well with acceptable error for basic medical data.

\subsection{Actual vs Predicted Comparison}
Figure 4 compares actual and predicted HC values. Points close to the red diagonal line indicate good predictions. The model predicts well for medium HC values (150-200mm), but has larger errors for extreme values.

\begin{figure}[h]
\centering
\includegraphics[width=0.4\textwidth]{figure/fig_scatter}
\caption{Comparison of actual and predicted HC on validation set}
\end{figure}

\subsection{Test Set Predictions}
The model successfully predicted HC for 335 images in the test set. Results are saved in \texttt{test\_predictions.csv}.

% ============================================
\section{Conclusion}
% ============================================

In this lab, I have:

\begin{enumerate}
    \item Explored fetal ultrasound image data
    \item Built a CNN model to predict head circumference
    \item Achieved MAE of about 49mm on validation set
    \item Applied the model to predict on test set
\end{enumerate}

\end{document}